\section{Introduction}
Substructure search in chemical compound databases is a vital task in cheminformatics, underpinning broad applications in drug discovery, 
materials science, and toxicology. The objective is to identify all molecules in a database that contain a given query substructure, which 
typically corresponds to a specific chemical motif or functional group. This search has been a cornerstone in understanding the influence 
of specific substructures on a compound's biological activity, physicochemical properties, and reactivity, a recognized concept for 
decades \cite{barnard1993}.

Historically, computer-based substructure search started with pioneers like Ledley and colleagues who developed the Chemical Substructure 
Search (CSS) algorithm in the 1960s and 1970s \cite{ledley1964}. This algorithm employed a graph-based approach to identify specific 
substructures in chemical compounds. Further advancements in the field came with the development of the Simplified Molecular Input 
Line Entry System (SMILES) notation by Weininger in the 1980s \cite{weininger1988}, which provided a simple, linear representation 
of molecular structures as strings.

Substructure search fundamentally rests on the solution of the subgraph isomorphism problem, a problem known to be NP-complete \cite{ullmann1976}. 
Due to its high computational complexity, numerous heuristics and algorithms have been devised to accelerate the search process. 
Among these, the Filter-and-Verification paradigm has been a prevalent approach, involving an initial filtering step to quickly 
eliminate unsuitable candidate graphs, and a more computationally intensive verification step to confirm the presence of the query 
substructure in the remaining candidates \cite{cordella2004, shasha2002}. Over time, graph-based subgraph isomorphism algorithms, 
such as the Ullmann algorithm \cite{ullmann1976} and the VF2 algorithm \cite{cordella2004}, have emerged as more efficient and 
scalable solutions for substructure search in large chemical databases.

In addition to these, frequent subgraph mining algorithms like gSpan \cite{yan2002}, FFSM \cite{kuramochi2001}, and Gaston \cite{nijssen2004} 
have proven valuable in identifying frequently occurring substructures in large sets of chemical compounds. These approaches 
are particularly beneficial for applications such as structure-activity relationship (SAR) analysis and molecular classification.

Efficient filtering techniques often involve the use of binary and quantitative features, or fingerprints, to represent molecular 
structures. These fingerprints facilitate the rapid elimination of graphs that do not contain specific features required by the 
query subgraph, thereby speeding up the substructure search process \cite{bonchi2011, klein2014}.

However, as the number of known molecules and the size of chemical databases have grown significantly, the traditional approaches, 
which often require full or nearly full enumeration of candidates, have become increasingly challenging to implement efficiently. 
This development underlines the need for more scalable solutions.

Our work introduces a unique approach aimed at mitigating these challenges. While in certain cases the algorithm may resort to exhaustive enumeration, in most scenarios it employs a more sophisticated strategy, transcending the conventional full enumeration paradigm. Instead, we introduce a unique index structure: a tree that segments the molecular dataset into clusters based on the presence or 
absence of features. Inspired by the binary Ball-Tree concept \cite{omohundro1989, clarkson1994}, this structure demonstrates 
superior performance over exhaustive search on average, leading to a significant acceleration in the filtering process.

We provide a comparative analysis of our method based on Bingo \cite{Pavlov2010} with the original Bingo algorithm, highlighting key differences. 

While Bingo uses advanced filtering effectively, it relies on exhaustive search in a chemical space that continually expands. In contrast, 
our approach departs from exhaustive search and places an existing molecular fingerprint into a tree structure, rather than a 
conventional relational database. While our current version does not impact the verification stage, it speeds up the filtering stage.
By introducing this innovative structure, we aim to cater to the growing scale of chemical databases and the escalating demand 
for efficient and scalable search solutions. Our approach offers potential for future research and application in the quest 
for more efficient and accurate substructure search techniques.
