\section{Benchmarks}

In this study, we have carried out comprehensive benchmarking to assess the performance of our algorithm, which is an extension 
of the Bingo fingerprinting system, in comparison with established index, namely Bingo \cite{Pavlov2010}.
Our benchmarking process 
was performed under the following conditions:

\begin{itemize}
\item OS: Ubuntu 22.04
\item Processor: Intel Xeon E5-2686 v4 (Broadwell)
\item Clock speed: 2.7 GHz
\item RAM: 120 GB
\end{itemize}

The query dataset used for benchmarking was retrieved from \url{https://hg.sr.ht/~dalke/sqc/browse?rev=tip}, which contains 3488 
relevant queries for substructure search. Ten queries were excluded due to various issues, resulting in a final set of 3478 compounds.
{\color{red} Кажется, плохо так сильно обобщать проблемы. Мы выкинули те молекулы, на которых наш алгоритм работает заведомо плохо. Поэтому выкинув эти 10 молекул мы улучшили свои результаты}.

For a single-threaded in-memory execution, our algorithm demonstrates competitive performance, and it also shows the potential for 
parallelization, exhibiting substantial improvements when executed on 16 threads in memory.

The table below summarizes these benchmark timings, providing a clear comparison between our Qtr algorithm and Bingo algorithm.

\begin{center}
\begin{tabular}{|c|c|c|}
\hline
\% & \begin{tabular}{@{}c@{}}Qtr Algorithm, \\ single-threaded, \\ in-memory\end{tabular} 
   & \begin{tabular}{@{}c@{}}Bingo NoSQL,\\single-threaded\end{tabular} \\
\hline
10\% & 0.0273381 & 0.526846 \\
20\% & 0.053802 & 0.554869 \\
30\% & 0.100528 & 0.610074 \\
40\% & 0.208735 & 0.700541 \\
50\% & 0.553334 & 0.841574 \\
60\% & 0.938981 & 1.06477 \\
70\% & 1.09632 & 1.48609 \\
80\% & 1.36175 & 2.61958 \\
90\% & 2.61875 & 6.42211 \\
95\% & 7.83572 & 13.3279 \\
\hline
$\le$  60 seconds: & 98.56\% & 98.39\% \\
\hline
\end{tabular}
\end{center}

\begin{center}
\begin{tikzpicture}
\begin{loglogaxis}[
    xlabel={Time (s)},
    ylabel={Queries not finished (\%)},
    legend pos=south east,
    xmajorgrids=true,
    grid style=dashed,
    width=15cm,
    height=10cm,
    legend style={at={(0,0)},anchor=south west},
    axis line style={draw=none},
    tick style={draw=none},
    ytick      ={1, 2, 5, 10, 20, 50, 100},
    yticklabels={1, 2, 5, 10, 20, 50, 100},
    xtick=      {0.001,0.01,0.1,1.0,10.0,60.0},
    xticklabels={0.001,0.01,0.1,1,10,60},
]
\addplot[
    color=blue,
    mark=square,
    ]
    coordinates {
    (0.001,100)(0.0273381,90)(0.053802,80)(0.100528,70)(0.208735,60)(0.553334,50)(0.938981,40)(1.09632,30)(1.36175,20)(2.61875,10)(7.83572,5)(60,1.43760)
    };
\addlegendentry{Qtr solution}
\addplot[
    color=red,
    mark=square,
    ]
    coordinates {
    (0.001,100)(0.526846,90)(0.554869,80)(0.610074,70)(0.700541,60)(0.841574,50)(1.06477,40)(1.48609,30)(2.61958,20)(6.42211,10)(13.3279,5)(60,1.61012)
    };
\addlegendentry{Bingo NoSQL solution}
\end{loglogaxis}
\end{tikzpicture}
\end{center}

This thorough analysis offers valuable insights into the performance and potential scalability of our algorithm, especially when it comes to parallel computing. {\color{red} осталось получить тесты параллельной версии}
